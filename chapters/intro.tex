\chapter{绪论}
\section{研究背景}

次级视皮层(V2)是皮层视觉通路上的第二个环节,从初级视皮层(V1)传出的信息,大部分经过V2送到其他纹状外视觉皮层。除了V1区的传入外,V2还接受丘脑枕核(pulvinar)的信息以及其他纹状外视觉皮层(包括背侧和腹侧的视觉通路)的反馈。所以,作为一个信息整合与反馈控制的枢纽,V2是视觉信息传递与处理过程中的重要一站\cite{01}。

此段是介绍论文的研究背景,建议写1.5页。并有3到5个左右的引用。



\section{国内外研究现状}
\subsection{XXX概述}


这一节,针对论文中设计的技术进行一个简要的发展历史说明。


\subsection{XXX介绍}

针对论文中使用的专业技术进行详尽的介绍。

\section{本论文的主要工作}

在论文绪论中介绍自己的主要工作,最好使用列表的形式。
根据上述研究内容和研究目标,本文主要的工作从以下几个方面展开:
\begin{enumerate}
\item 实现了一个XXX的模拟方案,在此基础上进行了相应的改进。
\item 由于XXX的局限性,从而提出了基于XXXX保存方法。
\item 采用XXX技术实现复杂的YYY。
\item 基于以上的理论研究和技术调研,实现了一个XXX。
\end{enumerate}
\section{本论文的组织结构}
本文主要分为七个章节,每个章节的具体内容安排如下:
\begin{enumerate}
\item 绪论:介绍XXX技术的发展,以及早期YYY的各种不足,XXX的出现和所解决的问题。
\item XXX的相关知识:介绍了使用XXX技术开发YYY所需要的相关知识。主要包括ZZZ方程的介绍、XXX技术的基本概念、YYY的坐标变化和XXX。
\item 需求分析:通过对XXX的调研,得出本系统开发和研究的必要性。
\item 概要设计:基于需求分析的基础,提出应用XXX技术合适的设计方案,包括XYZ。通过概要设计,为整个系统的实现做好铺垫。
\item 详细设计与实现:基于概要分析的设计方案,使用XXX技术去实现烟雾模拟系统。这部分主要是讲述每个设计过程中涉及的实现细节,通过对开发过程中难点的攻克,得以最终实现了一个XXX系统。
\item 系统测试与分析:在实现XXX系统之后,我们要对整个系统进行测评,主要是从功能性、稳定性、性能和视觉效果四个方面来分析,确保最终表现的渲染效果满足预期的结果。
\item 总结与展望:总结自己本文所完成的工作,以及对于XXX的未来发展的展望,并指出自己当前研究的不足之处有待日后改进。
\end{enumerate}

% 以下的代码确保余下的章节不会都在右边开始,导致左边的页面留白


\makeatletter
\@openrightfalse
\makeatother