\chapter{需求分析}
\section{选题的动机与目的}

介绍下为什么选择这个论文课题以及好处和优势。

我们选用XXX技术来开发的原因主要有以下两点:

\begin{enumerate}
\item 简化了开发成本:一次开发,多平台使用。和传统的客户端开发模式不同,XXX技术的跨平台性使得我们不需要针对特定的平台而进行额外的开发工作,用户只需要安装通用的浏览器即可正常使用。
\item 节约了系统的维护:因为我们所有的项目文件都是保存在服务器上,用户每次加载资源都需要从服务器上重新下载,所以在后期维护和更新项目的过程中,不再需要用户进行参与,直接通过服务器上的更新就可以很方便的完成升级和快速迭代开发。
\end{enumerate}

\section{选题的可行性分析}

经过本人的调研和研究,XXX技术通过完全可以胜任在YYY中实现复杂的ZZZ,并获得良好的实时效果,足以达到最终的需求。

\subsection{主要技术路线}
\begin{itemize}
  \item 应用架构:B/S架构
  \item 前台技术:HTML5
  \item 3D展现层:XXX
  \item 算法基础:Navier-Stokes方程
\end{itemize}

\subsection{核心技术关键}
\begin{itemize}
  \item 三维数据的操作与保存
  \item 求解Navier-Stokes方程
\end{itemize}

\section{本系统的需求概述}

一段话对当前的需求进行描述。

\section{本系统的功能性需求描述}

针对本论文实现的系统的具体功能点来列出每个功能的需求,最好使用用例图来进行描述。

\section{本系统的非功能性需求描述}

本系统的非功能性需求主要有以下四个方面:
\begin{enumerate}
\item 系统运行的效率
\item 系统运行的稳定性
\item 系统的可扩展性
\item 系统运行的真实性
\end{enumerate}
\section{本章小结}

除了绪论,每章的最后一节都必须有本章小结,这是学校的规定。